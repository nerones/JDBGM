\chapter{Diseño}
Acá un resumen del capitulo
\section{Bases}
Para diseñar JDBGM se tomo como idea base el patrón de desarrollo DAO, que básicamente sirve para abstraer y encapsular el acceso a la fuente de los datos, además DAO sera el encargado de manejar la conexión con la fuente de los datos para la obtención y guardado de los datos.
DAO esta pensada para encapsular el acceso a cualquier fuente de datos, es decir los datos podrían estar almacenados en un archivo de texto plano, una base de datos relacional, una base de datos no relacional o un archivo XML, pero en este caso se trabaja solo sobre motores de base de dato relacionales.  
\section{JDBC y su envoltorio}
y otro
\section{Abstracción de SQL}
Uno de los mayores problemas al intentar crear código SQL portable entre diferentes motores de bases de datos reside en que cada motor crea su propio dialecto de SQL. Las diferencias entre estos dialectos puede ser mínima pero aun así esto provoca que a una aplicación a la que se le quiera cambiar el motor de base de datos que usa se le deban revisar todas las sentencias SQL usadas.\\
Así que para brindar una abstracción sobre la base de datos es necesario que las consultas SQL (sin hacer diferencias sobre DDL \footnote{Data Definition Lenguaje} y DML \footnote{Data Manipulation Lenguaje}) no sean manejadas de forma explicita es decir que no seria conveniente, por ejemplo, manejar las consultas como cadenas de caracteres ya que esta es una estructura estática dependiente del DBMS que se este usando. Por lo que para poder manejar adecuadamente las consultas se considero usar una estructura de datos un poco mas compleja que contenga la consulta de manera desglosada. Es decir que en ves de tener una sentencia como la siguiente:
\begin{Verbatim}
  nombre_comando [parametro] opcion1 [parametro] opcion2 [parametro]; 
\end{Verbatim}

Se tendría una estructura de datos como la siguiente:

\lstset{frame=single, rulesepcolor=\color{black}, backgroundcolor=\color{lightgray}}
\begin{lstlisting}[language=java, numbers=left, numberstyle=\tiny, basicstyle=\footnotesize, title=Pseudocódigo de la estructura de dato que contiene la sentencia]
class Sentencia{
	nombre_comando;
	opcion1;
	opcion2;
	devolver_sentencia;
}
\end{lstlisting}

Hay que tener en cuenta que el tipo de datos que se quiere manejar aparte de contener los parámetros de la sentencia debe brindar métodos para armar la sentencia, asignarles valores a los atributos y ser capaz de definir el comportamiento especifico que pueda tener la sentencia. Cuando se dice que el tipo de dato, mas en concreto la clase, que represente la sentencia sea capaz de armar la sentencia se refiere a que sea capaz de devolver la cadena de caracteres que represente una sentencia SQL valida para algún DBMS.\\
En el caso especifico que se esta tratando se cubrirá un subconjunto de todas las sentencias posibles que define SQL. Este conjunto estará formado por:
\begin{enumerate}
\item CREATE TABLE - Permite crear tablas 
\item ALTER TABLE - Modifica la estructura de una tablas \footnote{Solo se da un soporte básico para esta sentencia}
\item UPDATE - Modifica las filas de una tabla 
\item INSERT - Inserta nuevas filas a una tabla
\item DELETE - Elimina columnas de una tabla
\item SELECT - Realiza consulta sobre las tablas
\end{enumerate}

Se agruparon las sentencias de acuerdo a el resultado que producen, por un lado tenemos a las sentencias que devuelven datos mas allá de una información que indique el éxito o fracaso de la interpretación de la sentencia en el DBMS o sea sentencias como SELECT, por otro lado tenemos aquellas que solo devuelven información del exito o fracaso de la interpretación de la sentencia.

\subsection{Diseño de CREATE TABLE}
Esta sentencia es usada para crear tablas en una base de datos relacional, un resumen de su sintaxis como se la define en el estándar SQL es la siguiente:
\begin{Verbatim}
  CREATE TABLE <table name> (
  { <column name> [ <column type> ][ PRYMARY KEY ][ REFERENCES <foreign table> ] }...
  )
  [ PRIMARY KEY <indexed columns> ]
  [ FOREIGN KEY <columns and referenced table> ]
\end{Verbatim}
Se obviaron las definiciones de algunas de las opciones de\verb= CREATE TABLE= puesto que escapan al alcance del proyecto el cual esta limitado por los DBMS a los que tiene que soportar además hay que tener en cuenta algunas salvedades, como que si se especifica una columna como\verb= PRIMARY KEY= dentro de la definicion de la columna no se puede especificar la clausula\verb= PRIMARY KEY= fuera de los paréntesis que encierran a las definiciones de las columnas. Mas específicamente podemos empezar con SQLite el cual, sin restarle importancia, es el que menos características implementa debido a su "pequeñez" por lo cual pasamos a estudiar \verb= CREATE TABLE= tal como la entiende SQLite \footnote{para ver la sintaxis usada ir a QQQQQQQQ} y comparando las diferencias con MySQL y PostgreSQL:

\begin{Verbatim}[frame=leftline, framesep=3mm]
  CREATE [ TEMP | TEMPORARY ] TABLE <database name> <dot> <table name> 
  [ IF NOT EXISTS ] <table contents source>

  <table contents source> ::=
    <left paren> <table element> [ { <comma> <table element> }... ] <right paren>
  | AS <select stmt>
\end{Verbatim}

Los parámetros \verb= TEMPORARY= y \verb= TEMP= se usan indistintamente para crear una tabla temporaria en SQLite, lo cual es soportados por los 3 DBMS con la salvedad que en MySQL solo se acepta la palabra reservada\verb= TEMPORARY=. El parámetro\verb= IF NOT EXISTS= indica que de existir una tabla del mismo nombre la sentencia \verb= CREATE TABLE= no tiene efecto sobre la base de datos, este parámetro no es soportado por PostgreSQL por lo que sera omitido. Por ultimo hay que tener en cuenta que al ser SQLite un motor "serverless" siempre se esta trabajando o sobre la base de dato principal o sobre una temporaria por lo que este para el parámetro\verb= <database name>= solo acepta \verb= "main"= ó \verb= "temp"= de modo que en un principio el nombre de la base de dato no se tomara en cuenta y cualquier tabla que se cree sera en la base de dato actual.

\begin{Verbatim}
  <table element> ::=
    <column definition>
  | <table constraint definition>

  <column definition> ::=
  <column name> [ <data type> | <domain name> ]
  [ DEFAULT <default option> ]
  [ <column constraint definition>... ]
  [ COLLATE <colation name> ]

  <default option>::=
    <signed number>
  | <literal value>
  | <left paren> <expr> <left paren>
  
  <column constraint definition> ::=
  [ CONSTRAINT <name> ] <column constraint>

  <column constraint> ::=
    NOT NULL <conflict cause>
  | UNIQUE <conflict cause>
  | PRIMARY KEY [ ASC | DESC ] <conflict cause> [ AUTOINCREMENT ]
  | <references specification>
  | CHECK <left paren> <expr> <right paren>
\end{Verbatim}

La siguiente sección a estudiar es la definición de columnas y restricciones de columna que se diferencian de las restricciones de tabla en que las primeras se definen junta a la definición de columna y solo se pueden hacer sobre una única columna, en cambio las restricciones de tabla se hacen después de las definiciones de columnas y estas pueden referirse a un grupo de columnas.\\
Como primer diferencia veremos las tipos de datos tienen diferentes nombres y significados, pero esto sera tratado mas adelante\footnote{referenciar a alguna parte de la tesis}. 

\begin{Verbatim}
  <table constraint definition> ::=
  [ CONSTRAINT <name> ] <table constraint>

  <table constraint> ::=
    [ UNIQUE | PRIMARY KEY ] <left paren> <unique column list> <right paren> <conflict cause>
  | <referential constraint definition>
  | CHECK <left paren> <expr> <right paren>

  <unique column list> ::= <column name list>

  <referential constraint definition> ::=
  FOREIGN KEY <left paren> <referencing columns> <right paren>
  <references specification>

  <references specification> ::=
  REFERENCES <referenced table and columns>
  [ MATCH <name> ] [ <referential triggered action> ]
  [ <constraint characteristics> ]

  <match type> ::=
  FULL
  | PARTIAL
  | SIMPLE

  <referencing columns> ::= <reference column list>

  <referenced table and columns> ::=
  <table name> [ <left paren> <reference column list> <right paren> ]

  <reference column list> ::= <column name list>

  <referential triggered action> ::=
    <update rule> [ <delete rule> ]
  | <delete rule> [ <update rule> ]

  <update rule> ::= ON UPDATE <referential action>

  <delete rule> ::= ON DELETE <referential action>

  <referential action> ::=
    CASCADE
  | SET NULL
  | SET DEFAULT
  | RESTRICT
  | NO ACTION

   <constraint characteristics> ::=
  [ NOT ] DEFERRABLE [ INITIALLY DEFERRED | INITIALLY IMMEDIATE ]
\end{Verbatim}

Así que finalmente tenemos las siguiente sintaxis para nuestro proyecto
\begin{Verbatim}
  CREATE [ TEMPORARY ] TABLE <database name> <dot> <table name> 
  <table contents source>

  <table contents source> ::=
    <left paren> <table element> [ { <comma> <table element> }... ] <right paren>
  | AS <select stmt>

\end{Verbatim}

\subsection{Diseño de UPDATE e INSERT}

\subsection{Diseño de ALTER}

\subsection{Diseño de DELETE}

\subsection{Diseño de SELECT}
