\chapter{Conclusiones}
%
%
Después de analizar, diseñar e implementar queda como ultima tarea la de exponer la experiencia que resulto de todo el trabajo echo, como introducción puedo destacar que tome verdadera conciencia de lo que implica analizar el problema antes de iniciar con un proyecto aun cuando este no sea muy grande, a la hora de diseñar e implementar dada mi poca o nula experiencia la tarea estuvo mezclada pues algunos detalles de diseño debido a la naturaleza de el proyecto requerían mayor conocimientos sobre el lenguaje de los que inicialmente poseía, además claro de otros conceptos mucho mas abstractos como por ejemplo los de los patrones y de las buenas practicas a la hora de desarrollar o mejor dicho escribir código.
%
%
\section{Resultado de el proyecto}
%
El resultado del proyecto
%
\section{Conclusiones generales}    
%
El mayor problema al que me  encontré en el desarrollo del proyecto fue la dificultad al decidir como implementar todo lo que se diseño, es muy distinto escribir código que sera usado para escribir más código, que código que no sera directamente utilizado, por ejemplo se podrían no obviar las buenas practicas a la hora de implementar un sistema eminentemente visual, un programa cualquiera, pues el usuario no ve directamente el código para el esta todo bien si es que la interfaz responde o se adecua a sus necesidades pero a la hora de escribir una librería por ejemplo es la sintaxis que se usa, los métodos que se ponen a disponibilidad, la facilidad de uso que se da la que ponen en juego la utilidad o complacencia del usuario frente a el trabajo realizado. Es además el usuario de la librería un par con una mayo capacidad de critica.