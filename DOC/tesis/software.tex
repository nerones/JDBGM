\chapter{Software}
Con el objetivo de contar con una herramienta para la investigacin y
los experimentos sobre los clasificadores neuronales LIRA y PCNC, as como para el localizador y la creacin de las bases de datos, se crearon varios paquetes de software. El software se desarroll mediante un lenguaje de Programacin Orientado a Objetos (POO). Los lenguajes orientados a
objetos permiten contar con mdulos de software reutilizables,
flexibles y de fcil mantenimiento. Esto
hace posible que la investigacin y desarrollo sean ms eficientes en lo que a software se refieren
y pone a disposicin inmediata aquellos mdulos de software que han
sido probados con buenos resultados ya sea para un sistema completo,
un sistema alternativo o para realizar otro tipo de experimentos.

El software desarrollado en primera instancia se compone de cuatro mdulos: Optik, RNA,
Localizador e Interfaz. El mdulo Optik se encarga de la creacin de
bases de datos de imgenes, el mdulo RNA es la implementacin del
clasificador neuronal LIRA, el mdulo Localizador busca una pieza
determinada en una imagen y la Interfaz es el mdulo encargado de la
intercomunicacin entre los dems mdulos as como con el usuario. En
la Fig. \ref{OptikRNADiagramaABloques} se muestra un diagrama a bloques de estos mdulos as como
las interconexiones entre stos y los archivos que se manejan, tambin
se listan las funciones disponibles para el usuario. Se describen a
continuacin con ms detalle cada uno de los mdulos en los se har
referencia continua a la figura mencionada.

\begin{figure}[h]
\begin{center}
\includegraphics[
%natheight=1.760800in,
%natwidth=2.603900in,
%height=1.2756in,
%width=\textwidth
%]{figuras/OptikRNADiagramaABloquesV2_1.png}
\caption[Diagrama a bloques del software OptikRNA.]{Diagrama a bloques del software OptikRNA. Se muestran los cuatro mdulos que lo componen, los archivos utilizados y sus
   principales funciones as como las intercomunicaciones entre todos
   estos elementos.}\label{OptikRNADiagramaABloques}
\end{center}
\end{figure}

El software desarrollado posee muchas ventajas de la programacin orientada a objetos (POO). Se utiliz el lenguaje de programacin C++. En un inicio se us el Ambiente integral de desarrollo Borland$^{MR}$ y con ello algunos objetos predefinidos por Borland$^{MR}$ fueron utilizados para la manipulacin de imgenes y la creacin de las interfaces grficas.

Ms tarde se decidi trasladar el cdigo existente a C++ estndar por lo que se abandon el uso de dicho ambiente de desarrollo. El uso del C++ estndar posibilita que el cdigo pueda ser trasladado fcilmente a cualquier plataforma y a sistemas embebidos, lo anterior es especialmente importante para nuestro proyecto. Adems, los costos de desarrollo se reducen al no depender de software comercial haciendo que el sistema sea ms econmico, caracterstica que es una de las pautas ms importantes en todo el trabajo desarrollado. Bajo esta nueva pauta de trabajo se desarroll tambin el software PCNC.

La aplicacin Optik est siendo desarrollado para GNU/Linux [REF] y las dems aplicaciones para el clasificador LIRA y el PCNC se desarrollaron en C++ estndar en este mismo sistema operativo. Tambin el localizador de piezas. Si bien no se les desarroll interfaz grfica, esto se hizo premeditadamente debido a las siguientes razones: 

\begin{enumerate} 
\item Las buenas prcticas de programacin moderna ensean que la interfaz grfica debe separarse totalmente de la implementacin del software particular. 
\item El poder que ofrece el hecho de poder correr el software desde la lnea de comandos cuyo ejemplo ms prctico se da en los archivos de procesamiento por lotes que posibilitaron la ejecucin de decenas de experimentos en una sola orden y 
\item La visin de que en algn momento del desarrollo futuro el sistema pueda ser embebido a un microcontrolador o algn otro hardware especializado.
\end{enumerate} 

\section{Mdulo Optik}\label{sec:optik}
Optik es un software que genera bases de datos de imgenes de objetos
destinadas para entrenamiento y pruebas del clasificador neuronal
LIRA. Se parte de imgenes generales de mltiples objetos ordenados
aleatoriamente, con ayuda de marcas colocadas por el usuario, genera
una base de datos de imgenes normalizadas mediante un proceso de
extraccin. imgenes normalizadas se refiere a que stas tienen
iguales caractersticas: contienen una pieza centrada y con
orientacin fija de 0 respecto a su eje mayor, son de dimensin
constante y tienen una ventana circular que facilita la rotacin
(Fig. \ref{optikpantallaconmuestrasetiquetasymarcas}). Las imgenes extradas son nombradas de acuerdo al tipo de
pieza correspondiente y a un nmero consecutivo. Tambin el sistema
crea un archivo MRK que contiene los datos de todas las marcas
realizadas.

\begin{figure}[h]
\begin{center} \includegraphics[width=10cm]
%{figuras/optikPantallaConMuestrasEtiquetasYMarcas.jpg}\end{center}
\caption[Interfaz grfica de usuario (IGU) de Optik.]{IGU Optik, la cual permite marcar y extraer las muestras de imgenes a partir de las imgenes escena.}
\label{optikpantallaconmuestrasetiquetasymarcas}
\end{figure}

Antes de el proceso de colocacin de muestras, deben definirse los
nombres y otras propiedades de los distintos tipos de piezas con que
se va a trabajar (i. e. dimensiones, peso y material), stos datos son
de utilidad al sistema para calcular automticamente el centro de la
pieza adems de futuras operaciones. Esta informacin se almacena en
el sistema en un archivo PZA y puede ser cargada cuando se
requiera. Optik adems realiza algunas tareas de preprocesamiento de
imgenes, como extraccin de contornos y conversin de imgenes de
color a escala de grises.

Debido al hecho de que las marcas en las escenas son puestas por el usuario mediante el ratn, las imgenes muestras para las bases de datos no estn centradas ni orientadas exactamente. Este no es un problema ya que los clasificadores utilizados son capaces de llevar a cabo la tarea de reconocimiento con muestras imperfectamente centradas o giradas, sobre todo en conjuntos grandes de entrenamiento.

Este software en primera instancia se elabor con Borland$^{MR}$ C++, luego se decidi en colaboracin con otros colegas pasarlo a sistema operativo GNU/Linux \cite{Josue2006}.

\section{Mdulo RNA}
El mdulo RNA implementa al clasificador neuronal LIRA, es decir, es
la realizacin de la red neuronal completa, neurona a neurona, sus
interconexiones y toda su funcionalidad. Este es el mdulo ms
elaborado, su eficiencia es crtica, por lo que fue necesario cuidar
dos aspectos fundamentales, memoria y velocidad. Un clasificador
neuronal puede necesitar cientos de miles de neuronas y una magnitud
mayor de interconexiones adems de ser indispensable tiempos de
ejecucin total aceptables, tanto en las fases de entrenamiento como
en las de prueba, siendo crtico en una aplicacin real de
reconocimiento (mdulo Localizador). Para cuidar la velocidad se
decidi programar este mdulo y por extensin todo OptikRNA con
lenguaje C++, es decir, se utiliz la velocidad y el poder de C
combinado con las ventajas de la POO. Se utilizaron punteros, estos
permiten operar a bajo nivel mejorando la velocidad y realizar las
interconexiones entre las distintas clases utilizadas ms
eficientemente. Para cuidar la eficiencia evitando operaciones de
punto flotante se utilizaron nicamente nmeros enteros. Este mdulo
se constituye de diversas clases, las principales, junto con su
relacin de herencia se muestran en la Fig. \ref{RNAclasesYHerencia}.

\begin{figure}[h]
\begin{center}
\includegraphics[
%natheight=1.760800in,
%natwidth=2.603900in,
%height=1.2756in,
width=4in]
%{figuras/RNAclasesYHerencia.png}
\caption[Clases principales del mdulo RNA y su relacin de herencia.]{Clases principales del mdulo RNA y su relacin de herencia. a) Clases a nivel neurona. b) Clases a nivel capa.}\label{RNAclasesYHerencia}
\end{center}
\end{figure}

Existe una sper clase llamada RNA la cual construye con las clases
descritas antes y otras menores no mencionadas el clasificador
neuronal LIRA. Esta clase ofrece las mismas funciones que tiene el
clasificador Lira descritas en la Seccin 2 adems de otras necesarias
para su implementacin y funcionamiento, entre las principales estn:
creacin, codificacin, entrenamiento, reconocimiento y borrado de
memoria (pesos a cero). La clase RNA controla completamente a todo el
mdulo y es con la cul se comunica realmente el mdulo Interfaz. 

\section{Mdulo Localizador}
El mdulo Localizador se encarga de buscar una pieza requerida en una
imagen fuente y definir la posicin de sta. La pieza que ser capaz
de localizar este mdulo debe estar en el conjunto de piezas con que
el clasificador se haya entrenado previamente. La imagen fuente no
tiene por que ser una imagen ocupada para la extraccin de imgenes
normalizadas, puede ser cualquier imagen siempre que contenga el
objeto a buscar en la misma escala en que existe en las imgenes
ocupadas.

Sobre una imagen dada, este mdulo aplica un algoritmo de bsqueda que
consiste en ir tomando subimgenes empezando por el centro y
desplazndose en forma espiral (Fig. 5b). Para cada posicin se pide
al mdulo RNA identificar la pieza requerida y si no se encuentra se
rota un cierto ngulo y se repite el proceso hasta encontrar la pieza
o cuando una rotacin es completada, si no se encuentra, se continua
el desplazamiento espiral hasta encontrar la pieza buscada o alcanzar
los lmites de la imagen. Los desplazamientos lineales y angulares
pueden ser definidos por el usuario.

El mdulo Localizador constituye una aplicacin prctica concreta en
microensamble y puede funcionar independientemente de los mdulos
Optik e Interfaz con el fin de ser acoplado a sistemas automticos de
manipulacin de piezas. En la Fig. 5c se muestra un resultado de la
bsqueda de una pieza ("tornillo de cabeza redonda") sobre una imagen
que contiene diversas piezas. El sistema despliega las coordenadas de
la pieza as como la orientacin. Cuando la pieza se localiza lejos de
su centro, la orientacin es errnea, por esta razn este mdulo debe ser mejorado.

\section{Interfaz}
El mdulo Interfaz es el nico mdulo que se comunica con el usuario,
adems intercomunica los dems mdulos para administrarlos. Este
mdulo en si es una Interfaz Grfica de Usuario (IGU), cuenta con
todas las funciones disponibles de OptikRNA y con reas para desplegar
imgenes y resultados. En la Fig. \ref{redneuartpantallapruebaeinfo}
se muestra sta interfaz. El rea llamada "parmetros del clasificador" es a donde el
usuario ingresa los parmetros para la creacin de un clasificador
neuronal LIRA en particular. Abajo est el panel de funciones bsicas
de la RNA, desde aqu se envan los comandos para el mdulo RNA, como
crear, guardar, cargar y reconocer. En el rea de mensajes se
despliega informacin general del clasificador cargado. En el panel de
funciones avanzadas existen funciones para el mdulo RNA como
entrenar, probar, asignar bases de datos y entrenar-probar
automticamente. En el rea de resultados se muestran los resultados
de las pruebas o el reconocimiento de piezas. Por ltimo, desde el
panel del mdulo Localizador accesamos a las funciones y parmetros de
este mdulo. En la figura referida se muestra un clasificador cargado
junto con su informacin general y los resultados de una prueba
aplicada. En el rea vaca es donde se despliega la imagen utilizada
por el Localizador. Las funciones del mdulo Optik estn en otra IGU que no se muestra.

\begin{figure}[h]
%\begin{center} \includegraphics[width=15cm]{figuras/redneuartPantallaPruebaEInfoV2.jpg}\end{center}
\caption{Interfaz grfica de usuario del software RNA.}
\label{redneuartpantallapruebaeinfo}
\end{figure}


\begin{figure}[h]
\begin{center}
%\includegraphics[natheight=1.760800in,natwidth=2.603900in,height=1.2756in,width=1.8784in]{figuras/blockDiagramRedneuartOptikV2.jpg}
\caption{Diagrama a bloques del software OptikRNA.}\label{blockdiagramredneuartoptik}
\end{center}
\end{figure}

El usuario trabaja con el mdulo de software de redes neuronales que implementa el clasificador LIRA, esto lo hace a travs de la IGU Rna (Fig. \ref{redneuartpantallapruebaeinfo}). Con la ayuda de esta interfaz el usuario puede crear el clasificador neuronal LIRA, entrenar, probar y usar el clasificador as como utilizar el localizador de piezas. La IGU Rna con un clasificador entrenado ya cargado es mostrada en la Fig. \ref{redneuartpantallapruebaeinfo}. En el cuadro de texto de la izquierda se muestra informacin sobre el clasificador. Los resultados se dan en el cuadro de texto de la derecha: el porcentaje de reconocimiento obtenido y los nombres de los archivos de las muestras que el sistema no pudo reconocer.

En el centro de la Fig. \ref{redneuartpantallabusquedaehistograma} se muestra una escena en la cul se ha llevado a cabo el procesamiento con el localizador de piezas. Un clasificador previamente entrenado y probado ha sido ya cargado. En esta misma imagen se muestra tambin en el cuadro de texto de la derecha, una marca sobre la pieza localizada as como las coordenadas y orientacin asociada a la misma.

\begin{figure}[h]
\begin{center} \includegraphics[width=15cm
]{figuras/redneuartPantallaBusquedaEHistograma.jpg}\end{center}
\caption{IGU Rna. Localizador de piezas.}
\label{redneuartpantallabusquedaehistograma}
\end{figure}

\section{Implementacin de LIRA}
Our neural classifier software is based on integer numbers operations in order
to avoid large time effort that is necessary for floating point operations.
The most general classes were made, that means that the same software modules
can be used to create different neural network topologies. The user
interface was made specially for LIRA neural classifier. It
is not hard to use this software modules to construct other types
of neural networks or make changes to the current topology, e. g.
add more layers. 

The most important classes created for artificial neural network realization
were: Dentrita, Neuron, NeuralSet and Rna.

The Dentrita class is a basic one. It has only two attributes, stimulus
and weight. Dentrita instances are used widely by neuron objects in
order to create fully functional neurons. In Fig. \ref{classneuron}
the Neuron class and its derived classes are shown. The neuron types used in LIRA classifier
are ON, OFF, AND and adding neurons. The ON and OFF neurons are one
input (OI) neurons and the rest are several input (SI) neurons. The
Neuron class is the fundamental part of the neural network software.

\begin{figure}[h]
\begin{center} \includegraphics[
natheight=2.000300in,
natwidth=4.125200in,
height=2.0384in,
width=4.1753in
%]{figuras/classNeuron.jpg}\end{center}
\caption{Clase neurona y sus clases derivadas.}
\label{classneuron}
\end{figure}

In order to construct neuron sets a general NeuralSet class was
created. Rna is the class that combines all mentioned classes. This class we use to
construct a LIRA neural network. Examples of the parameters
are the number of neurons or groups in each layer, the number of ON
and OFF neurons in each group, the input vector and the output classes.
The Rna object contains information about itself. It is possible to store
the complete neural network including all its internal parameters,
to load a previously stored neural network and to perform training
and recognition processes. 

The neural classifier is controlled by an object called
Rna-Interface. That interrelates the user by means the GUI with the databases used for training and testing of the classifier. 

\section{Software de lnea de comandos}
\subsection{lira2007}
A continuacin se presenta la ayuda propia del software \emph{lira2007} que ha sido usado para implementar y experimentar con el clasificador LIRA as como su interaccin con las bases de datos.

\begin{verbatim}
Example to use follow()-functions:

USAGE:
--help, -h
       display this help.

*** Setup ***
--create adn='<RNA name (string)> <tamVectEnt> <ImageWidth> <windowWidth> <windo
wHeight> <numGroup> <elemXGroup> <numNeuON> <numNeuOFF> <eta (double)> <numClass
es>', you can use -c instead of --create. Non especified are (int).
For example use: 
	 lira -c adn='liraSUN 22500 150 15 15 170000 7 4 3 1.0 8'  
       Create a Lira Neural Classifier from scrath.
--load <filename.rna>, -l <filename.rna>
       load a neural classifier from specified file.
--info, -i
       displays info about the loaded classifier.
--classes-file <filename.ent>, -cf <filename.ent>
       assign the classes file to LIRA classifier.

*** Preparing ***
--train-dir <dir>, -td <dir>
       assign the training directory, default is "./train".
--code-dir <dir>, -cd <dir>
       assign the code directory, default is "./code".
--code, -k
       code all the images in the train directory, they will put in the code di
rectory
--reset, -kill
       reset the internal connections of the loaded ANN, Erase its memory
--train num, -t num
       train using the coded images from the given code dir using "num" cycles.
 Default is 40
--prove-dir <dir>, -pd <dir>
       assign the prove directory, default is "./prove".
--images-dir <dir>, -id <dir>
       assign the images directory, default is "./images".
--train-percentage num, -tp num
       Percentaje of images in images-dir to be selected for the training set. 
Default is 50%

*** Using ***
--prove, -p
       proves the classifier with the images files stored in the "proving direc
tory".
--recognize <filename.png>, -r <filename.png>
       recognize a class in the given normalized image file.
pos='<x (int)> <y (int)> <O (double)>', Parameters especification for recogniz
ed a big (not normalized) image in certain point and orientation. All values s
hould be (int).
You might specified a "filename.png" by -r to look for.
Example use: 
	 lira -l lira.rna -r imagen.png pos='10 20 -45.0' 
An image file called "cut.png" will be created with the used subimage.

*** Ending***
--save, -s
       Save the current classifier to a .rna file.
--close, -q
       (NOW FAIL!) Save the current classifier to a .rna file.
       close the current classifier loaded in memory.
--output, -o
       Print values from the output layer.

*** Search ***
--searchImage <filename.png>, -si <filename.png>
       Search for some recognized clase.
--searchClass <className>, -sc <className>
       Class to be search by the Search command. Use "prueba" to make a prove
 and "cualquiera" to search anyone. Default is anyone
--searchQuantity num, -sq num
       Number of objects to search for. Default is 1.
--searchStep num, -ss num
       Step in pixels for the searcher to jump (/\x & /\y). Default is 20.
--searchStepAngle num, -sa num
       Angle in degrees for the searcher to jump. Default is 45.

*** Tools ***
--distortion dis='<dx> <nx> <dy> <ny> <dO> <nO>', you can use -d instead of -
-distortion. All parameters should be (int).
You might specified the "training dir".
For example use: 
	 lira -td ./trainimages -d dis='5 2 5 2 5 2' 
\end{verbatim}

\subsection{pcnc2007}
A continuacin se presenta la ayuda propia del software \emph{pcnc2007} que ha sido usado para implementar y experimentar con el PCNC as como su interaccin con las bases de datos.

\begin{verbatim}
Example to use follow()-functions:

USAGE:
--help, -h
       display this help.

*** Setup ***
--create adn='<PCNC name (string)> <method> <Tmin> <Tmax> <w> <h> <p> <n> <S>
 <N> <K> <Dc> <q> <numClasses>', you can use -c instead of --create. Non espe
cified are (int).
For example use: 
	 peco -c adn='PCNCprueba 0 1 32000 5 5 3 2 500 1000 12 8 5 8'  
       Create a Permutative Code Neural Classifier (PCNC) from scrath.
--load <filename.pcnc>, -l <filename.pcnc>
       load a PCNC from specified file.
--info, -i
       displays info about the loaded PCNC.
--classes-file <filename.ent>, -cf <filename.ent>
       assign the classes file to PCNC.

*** Preparing ***
--train-dir <dir>, -td <dir>
       assign the training directory, default is "./train".
--code-dir <dir>, -cd <dir>
       assign the code directory, default is "./code".
--code, -k
       code all the images in the train directory, they will put in the code 
directory
--reset
       reset the internal connections of the loaded PCNC, Erase its memory
--train num, -t num
       train using the coded images from the given code dir using "num" cycles
. Default is 40
--prove-dir <dir>, -pd <dir>
       assign the prove directory, default is "./prove".
--images-dir <dir>, -id <dir>
       assign the images directory, default is "./images".
--train-percentage num, -tp num
       Percentaje of images in images-dir to be selected for the training set.
 Default is 50%

*** Using ***
--prove, -p
       proves the PCNC with the images files stored in the "proving directory".
--recognize <filename.png>, -r <filename.png>
       recognize a class in the given normalized image file.

*** Ending***
--save, -s
       Save the current PCNC to a .pcnc file.
--close, -q
       (NOW FAIL!) Save the current PCNC to a .pcnc file.
       close the current PCNC loaded in memory.
\end{verbatim}

\subsection{Potencial para la experimentacin}
Como un ejemplo del uso ventajoso del software de lnea de comandos se presenta uno de los mltiples archivos de procesamiento por lotes\footnote{Ms conocidos por su denominacin en ingls: \emph{scripts}} empleado para realizar decenas de experimentos con cambi de parmetros de forma automtica, es decir, en una sola corrida.

\begin{verbatim}
#!/bin/bash

#Decenas de pruebas para estudiar comportamiento de parmetros
#La base es el Mejor:
#../../bin/pcnc2007 -c adn='PCNCmejorBD-A 1 0 65535 10 10 5 4 1000 300000 20 5
 2 8' -cf BD-A.ent -td ../../muestras/entrenamiento -cd ../../muestras/codigo 
-pd ../../muestras/prueba -k -t 40 -p -s


#*************PARAMETRO W
#Inicializacion
PARAM=W
PARw=10
PARp=5
PARn=4
PARS=1000
PARN=300000
PARK=20
PARDc=5
PARq=2
#Usando "for" para probar distintos parmetros 
for PARw in 05 08 09 11 12 15 20;
  do
  echo "**************************"
  echo "***** CORIDA: "$PARAM:$PARw" *******"
  echo "**************************"
  #Hace todo de una vez, tambin salva
  ../../bin/pcnc2007 -c adn='PCNCparam'$PARAM''$PARw' 1 0 65535 '$PARw' '$PARw
' '$PARp' '$PARn' '$PARS' '$PARN' '$PARK' '$PARDc' '$PARq' 7' -cf BD-D.ent -td
 ../../muestrasBD-D/entrenamiento -cd ../../muestrasBD-D/codigo -pd ../../mues
trasBD-D/prueba -k -t 30 -p -s

# ../../bin/pcnc2007 -c adn='PCNCparam'$PARAM' 1 0 65535 '$PARw' '$PARw' '$PAR
p' '$PARn' '$PARS' '$PARN' '$PARK' '$PARDc' '$PARq' 7' -cf BD-D.ent -i
done  

#*************PARAMETRO K
#Inicializacion
PARAM=K
PARw=10
PARp=5
PARn=4
PARS=1000
PARN=300000
PARK=20
PARDc=5
PARq=2
#Usando "for" para probar distintos parmetros 
for PARK in 05 10 15 25 30;
  do
  echo "**************************"
  echo "***** CORIDA: "$PARAM:$PARK" *******"
  echo "**************************"
  #Hace todo de una vez, tambin salva
  ../../bin/pcnc2007 -c adn='PCNCparam'$PARAM''$PARK' 1 0 65535 '$PARw' '$PARw
' '$PARp' '$PARn' '$PARS' '$PARN' '$PARK' '$PARDc' '$PARq' 7' -cf BD-D.ent -td
 ../../muestrasBD-D/entrenamiento -cd ../../muestrasBD-D/codigo -pd ../../mues
trasBD-D/prueba -k -t 30 -p -s
done  

#*************PARAMETRO Dc
#Inicializacion
PARAM=Dc
PARw=10
PARp=5
PARn=4
PARS=1000
PARN=300000
PARK=20
PARDc=5
PARq=2
#Usando "for" para probar distintos parmetros 
for PARDc in 02 04 06 08 10 15 20 25;
  do
  echo "**************************"
  echo "***** CORIDA: "$PARAM:$PARDc" *******"
  echo "**************************"
  #Hace todo de una vez, tambin salva
  ../../bin/pcnc2007 -c adn='PCNCparam'$PARAM''$PARDc' 1 0 65535 '$PARw' '$PARw
' '$PARp' '$PARn' '$PARS' '$PARN' '$PARK' '$PARDc' '$PARq' 7' -cf BD-D.ent -td
 ../../muestrasBD-D/entrenamiento -cd ../../muestrasBD-D/codigo -pd ../../mues
trasBD-D/prueba -k -t 30 -p -s
done  

#*************PARAMETRO q
#Inicializacion
PARAM=q
PARw=10
PARp=5
PARn=4
PARS=1000
PARN=300000
PARK=20
PARDc=5
PARq=2
#Usando "for" para probar distintos parmetros 
for PARq in 00 01 03 04 05 10;
  do
  echo "**************************"
  echo "***** CORIDA: "$PARAM:$PARq" *******"
  echo "**************************"
  #Hace todo de una vez, tambin salva
  ../../bin/pcnc2007 -c adn='PCNCparam'$PARAM''$PARq' 1 0 65535 '$PARw' '$PARw
' '$PARp' '$PARn' '$PARS' '$PARN' '$PARK' '$PARDc' '$PARq' 7' -cf BD-D.ent -td
 ../../muestrasBD-D/entrenamiento -cd ../../muestrasBD-D/codigo -pd ../../mues
trasBD-D/prueba -k -t 30 -p -s
done  

#*************PARAMETROS p y q
#Inicializacion
PARAM=PyN
PARw=10
PARp=5
PARn=4
PARS=1000
PARN=300000
PARK=20
PARDc=5
PARq=2
#Usando "for" para probar distintos parmetros 
for PARpn in "3 6" "4 5" "6 3" "7 2" "2 5" "3 4" "4 3" "5 2" "6 1" "4 7" "5 6"
 "6 5" "7 4" "8 3" "9 2";
  do
  echo "**************************"
  echo "***** CORIDA: "$PARAM:$PARpn" *******"
  echo "**************************"
  #Variable auxiliar para nombrar sin espacios
  PARpnNom=${PARpn/ /y}
  #Hace todo de una vez, tambin salva
  ../../bin/pcnc2007 -c adn='PCNCparam'$PARAM''$PARpnNom' 1 0 65535 '$PARw' '$P
ARw' '"$PARpn"' '$PARS' '$PARN' '$PARK' '$PARDc' '$PARq' 7' -cf BD-D.ent -i -td
 ../../muestrasBD-D/entrenamiento -cd ../../muestrasBD-D/codigo -pd ../../mues
trasBD-D/prueba -k -t 30 -p -s
done  



#*************PARAMETRO N
#Inicializacion
PARAM=N
PARw=10
PARp=5
PARn=4
PARS=1000
PARN=300000
PARK=20
PARDc=5
PARq=2
#Usando "for" para probar distintos parmetros 
for PARN in 100000 200000 250000 400000 500000;
  do
  echo "**************************"
  echo "***** CORIDA: "$PARAM:$PARN" *******"
  echo "**************************"
  #Hace todo de una vez, tambin salva
  ../../bin/pcnc2007 -c adn='PCNCparam'$PARAM''$PARN' 1 0 65535 '$PARw' '$PARw
' '$PARp' '$PARn' '$PARS' '$PARN' '$PARK' '$PARDc' '$PARq' 7' -cf BD-D.ent -td
 ../../muestrasBD-D/entrenamiento -cd ../../muestrasBD-D/codigo -pd ../../mues
trasBD-D/prueba -k -t 30 -p -s
done  


#*************PARAMETRO S
#Inicializacion
PARAM=S
PARw=10
PARp=5
PARn=4
PARS=1000
PARN=300000
PARK=20
PARDc=5
PARq=2
#Usando "for" para probar distintos parmetros 
for PARS in 0400 0600 0800 1200 1500 2000 2500 3000 4000 5000 6000 10000;
  do
  echo "**************************"
  echo "***** CORIDA: "$PARAM:$PARS" *******"
  echo "**************************"
  #Hace todo de una vez, tambin salva
  ../../bin/pcnc2007 -c adn='PCNCparam'$PARAM''$PARS' 1 0 65535 '$PARw' '$PARw
' '$PARp' '$PARn' '$PARS' '$PARN' '$PARK' '$PARDc' '$PARq' 7' -cf BD-D.ent -td
 ../../muestrasBD-D/entrenamiento -cd ../../muestrasBD-D/codigo -pd ../../mues
trasBD-D/prueba -k -t 30 -p -s
done  

#Crea el resumen de resultados
grep "El porcentaje" PCNCparam*.proveresults >> resumen.proveresults

#FIN

#  LocalWords:  PARw
 
\end{verbatim}

%%% Local Variables: 
%%% mode: latex
%%% TeX-master: "tesis"
%%% End: 
