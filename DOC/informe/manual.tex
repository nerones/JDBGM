\chapter{Manual de usuario}
\jj es sencillo de utilizar, para empezar es necesario contar con \jd que usualmente esta disponible por defecto en el SDK de JSE\footnote{Y por supuesto también en JEE.} y el driver de el motor que se pretenda utilizar aparte claro de contar con \jj. Una vez que se tienen a disposición las dependencias nombradas es necesario hacerle conocer al compilador la ubicación de los archivos necesarios\footnote{No se dará una explicación de como hacer esto pues no es la intención del manual enseñar el uso de Java.}.
%
\section{Primeros pasos}
Para empezar a usar el paquete es necesario importar algunas librerías, el siguiente seria un ejemplo típico de las librerías necesarias.
%
\begin{lstlisting}[title=Librerías a importar]
import java.sql.ResultSet;
import java.sql.Types;

import com.crossdb.sql.Column;
import com.crossdb.sql.CreateTableQuery;
import com.crossdb.sql.InsertQuery;
import com.crossdb.sql.SQLFactory;
import com.crossdb.sql.SelectQuery;
import com.nelsonx.jdbgm.GenericManager;
import com.nelsonx.jdbgm.JDException;
import com.nelsonx.jdbgm.ManagerFactory;
\end{lstlisting}
%
Como se ve se esta importando por un lado a el manejador de sentencias que se ubican dentro del paquete \verb=com.crossdb=, la capa de abstracción dentro del paquete \verb=com.nelsonx.jdbgm= y por ultimo algunas clases de \jd. Esto se podría hacer de una manera resumida utilizando el símbolo asterisco como se ilustra a continuación:
%
\begin{lstlisting}[title=Librerías a importar forma resumida]
import java.sql.ResultSet;
import java.sql.Types;

import com.crossdb.sql.*;
import com.nelsonx.jdbgm.*;
\end{lstlisting}
%
Pero de este modo se importan todas las clases incluidas dentro de un paquete en particular, lo que puede significar que se estén importando clases que no sean necesarias, de todos modos esta elección queda en manos del programador. Un detalle importante a notar es que en ningún momento se están importando los paquetes con las implementaciones especificas de las sentencias, esto debido a que se esta usando el patrón \textit{Abstract Factory} el cual hace transparente la elección de las implementaciones especificas de las sentencias, a modo informativo se señala que las implementaciones especificas de las sentencias se encuentran en los siguientes paquetes:
%
\begin{itemize}
\item Para \p en \verb=com.nelsonx.postgre=
\item Para \m en \verb=com.spaceprogram.sql.mysql=
\item Para \s en \verb=com.nelsonx.sqlite=
\end{itemize} 

A continuación lo que se debe hacer es registrar el motor que se quiere utilizar tal como se ilustra a continuación:	
%
\begin{lstlisting}[title=Registrando el motor que se quiere utilizar.]
public void test() throws JDException{
		
		GenericManager manager = ManagerFactory.getManager(
				ManagerFactory.MYSQL_DB
				, "user"
				, "localhost/test"
				, "password");
		SQLFactory sentencesFactory = ManagerFactory.getSQLFactory();
}
\end{lstlisting}
%
El método \verb=ManagerFactory.getManager= es el que precisamente sirve para registra el motor y obtener una instancia de el manejador del mismo, asignado a la variable \verb=manager= en este caso, los parámetros que recibe son:
\begin{itemize}
\item Una constante que identifica el motor que se quiere registrar y cuyos valores posibles son \verb=ManagerFactory.MYSQL_DB, ManagerFactory.SQLITE_DB= y por ultimo \verb=ManagerFactory.POSTGRE_DB=. De agregarse soporte para mas motores se deben agregar sendas constantes identificativas.
\item Un nombre de usuario con acceso a la base de datos.
\item Una URI hacia la ubicación de la base de datos.
\item La contraseña del usuario que se le proporciono en el parámetro anterior.
\end{itemize}
De ser los parámetros correctos ya se podrá empezar a interactuar con la base de datos señalada por la \verb=URI= que se le dio al método. Es importante notar que el método que se esta usando de ejemplo relanzara los errores que puedan ocurrir en cualquiera de los métodos que se están usando de \jj, de otro modo se tendría que haber utilizado los bloques \verb=try/catch= para manejar los mismos.

Como ultimo paso antes de poder empezar a trabajar sobre el motor es necesario obtener una fabrica de sentencias, para lo cual se utiliza el método \verb=getSQLFactory= que es un ``acceso directo'' al método de clase que brinda \verb=SQLFactory=.
%
\section{Construyendo sentencias}
%
%
Una vez obtenida la fabrica de sentencias, las sentencias pueden empezar a crearse a partir del objeto fabrica, siempre se debe utilizar las interfaces genéricas de las sentencias como tipos de datos de modo de que no se dependa de una implementación en especifico de la sentencia, esto se puede observar en el siguiente ejemplo:
%
\begin{lstlisting}[title=Uso de la fabrica de sentencias]
CreateTableQuery create = sentencesFactory.getCreateTableQuery();
InsertQuery insert = sentencesFactory.getInsertQuery();
UpdateQuery update = sentencesFactory.getUpdateQuery();
AlterTableQuery alter = sentencesFactory.getAlterTableQuery();
SelectQuery select = sentencesFactory.getSelectQuery();
\end{lstlisting}
%
En el extracto de código anterior se obtuvieron objetos que representan a cada una de las sentencias, a estas se las puede empezar a poblar de datos para después obtener las sentencias SQL que se deseen, a continuación un ejemplo para \verb=insert=:
%
\begin{lstlisting}[title=Uso de una sentencia INSERT]
InsertQuery insert = sentencesFactory.getInsertQuery();
insert.setTable("example_table");
insert.addColumn("columna_pk", 1);
insert.addColumn("columna_clave_foranea", 3);
manager.update(insert);
\end{lstlisting}
%
Como se ve la sintaxis ofrecida es bastante amigable con el lenguaje utilizado por SQL, para una completa referencia de los métodos disponibles para cada una de las sentencias es necesario referirse a la documentación incluida en el código fuente del proyecto. La sentencia formada con los métodos anteriores se corresponde con
\begin{Verbatim}
  INSERT INTO example_table 
  (columna_pk, columna_clave_foranea) 
  VALUES 
  (1, 3)
\end{Verbatim}
La cual es en la ultima porcion del código enviada a el motor mediante el método \verb=manager.update(insert)=, en este caso no se recupera el valor devuelto por dicha función que indica el numero de filas afectadas por la sentencia, en este caso seria $0$ por que insert agrega una fila, no altera ninguna. Este numero puede ser utilizado para saber si la consulta afecto o no a alguna fila para una consulta \verb=UPDATE= por ejemplo.
%
\begin{lstlisting}[title=Uso de una sentencia SELECT]
SelectQuery select = sentencesFactory.getSelectQuery();
select.addTable("example_table");
select.addColumn("column_pk");
select.addJoin().innerJoin("anoter_table", "example_table.id = anoter_table.id");
select.addOrderBy("column_pk");
ResultSet result = manager.query(select); 
\end{lstlisting}
%
El segmento de código anterior es para una sentencia \verb=SELECT= y la sentencia en particular que se formo es
\begin{Verbatim}
  SELECT column_pk
  FROM example_table INNER JOIN anoter_table
  ON example_table.id = anoter_table.id
  ORDER BY column_pk
\end{Verbatim}
Y al igual que en el ejemplo anterior se envía también la sentencia hacia el motor pero mediante el método \verb=query()= que es el apropiado para enviar las sentencias del tipo \verb=SELECT=, todas las otras sentencias se envían mediante \verb=update()=.

Siempre que se hace una consulta, su resultado debe ser obtenido a través de un objeto \verb=ResultSet=, este objeto es parte de \jd por lo que para una completa guía de su uso es necesario consultar la documentación\citep{java:jdbc:tutorial} de Oracle disponible en su web. Como ya se comento para una completa referencia de todos los métodos disponibles para cada una de las sentencias es necesario referirse a su documentación, la cual esta también disponible en  el código fuente de \jj en formato de \verb=javadoc=. No se profundiza mas en el uso de las sentencias puesto que los nombres de los métodos son bastante auto descriptivos.
%
\section{Realizando transacciones}
Independientemente de cuales sentencias se quieran enviar hacia el motor estas se pueden hacer de dos maneras, la normal que es la que se explico en la sección anterior y para ese caso si la  sentencia puede ser ejecutada, no tienen errores, los cambios que ella indique son permanentes, la segunda manera es hacerlo mediante transacciones. Una transacción trata un grupo de sentencias como si de una sola se tratase, de modo que si alguna de ellas falla o es imposible de realizar se revierten todos los cambios que se estaban por realizar, quedando la base de datos en el mismo estado que estaba antes de ejecutar la transacción, si ningún error ocurre recién todos los cambios que generaría la transacción son impactados permanentemente en la base de datos, la manera de realizar esto con \jj es de la siguiente:
%
\begin{lstlisting}[title=Realizando una transacción]
manager.beginTransaction();
	for (int i = 0; i < 10; i++) {
		manager.update(insert);
	}
manager.endTransaction();
\end{lstlisting}
%
Los métodos \verb=beginTransaction()= y \verb=endtransaction()= demarcan el inicio y fin de la , cualquier sentencia que se ejecute en medio de estas dos no impactara completamente hasta que la transacción sea finalizada, de ocurrir algún error toda la transacción, o lo que se había ejecutado de ella, sera desecha (\verb=rollback=) y además se detendrá la ejecución del código pues se lanzara una excepción. En el ejemplo, suponiendo que el objeto \verb=insert= represente una sentencia valida, supóngase que  algunos de los \verb=insert= enviado hacia el motor no pueda ser ejecutado sobre la base de datos entonces los cambios previos dentro de la transacción serán desechos y el ciclo interrumpido pues se lanzara una excepción que debe ser o  bien relanzada o capturada dentro de un bloque \verb=try/catch.= Los métodos \verb=GenericManager.queryAndClose()= y \verb=GenericManager.updateAndClose()= no se pueden usar dentro de una transacción puesto que estas trabajan independientemente de la única conexión  que maneja \verb=GenericManager= por lo que no se verán afectadas por la transacción.
%
\section{Liberando recursos}
Con \jj el único momento en que el programador se debe preocupar en liberar recursos se da en dos situaciones:
\begin{enumerate}
\item Cuando se esta obteniendo datos desde un \verb=ResultSet= puesto que este no se puede cerrar hasta que se terminen de obtener los datos del mismo y esto significa que hay un objeto \verb=Statement= consumiendo recursos por detrás que no puede ser cerrado. Por lo tanto cuando se terminen de obtener los resultados del mismo es necesario cerrarlo mediante el método \verb=ResultSet.close()=.
\item Cuando se considere que la base de datos ya no se requiera o bien que la base de datos no vaya a ser utilizada por bastante tiempo, se debe cerrar toda conexión con el motor mediante \verb=GenericManager.endConection()=.
\end{enumerate}

Con eso se cubren todos los aspectos básicos del uso de \jj, pero para empezar a usarlo se recomienda que antes se lea la documentación del mismo, al menos de las clases importantes como son \verb=GenericManager, ManagerFactory, SQLFactory= y todas las clases que definen a las sentencias.