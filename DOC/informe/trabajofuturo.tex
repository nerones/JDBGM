\chapter*{\addcontentsline{toc}{chapter}{Trabajo Futuro}Trabajo Futuro}
El resultado final de el trabajo contempla en mayor o menor medida todos los requisitos que se plantearon inicialmente, de todos modos existen aspectos a mejorar y corregir que no fueron encontrados o tenidos en cuenta en las pruebas que se realizaron, para ello en un futuro cuando la librería sea utilizada para desarrollar sera posible pulir mas las funcionalidades disponibles, o si es requerido implementar soporte a una mayor cantidad de motores, algo que en definitiva puede ser desarrollada por un tercero quedando como trabajo unificar ese trabajo con la distribución principal del proyecto.

En cuanto a la ampliación de funcionalidades puede ser importante implementar soporte para concurrencia en donde la librería se deba enfrentar a múltiples peticiones desde diferentes hilos (o incluso clientes en un modelo cliente servidor) entrando mas en el terreno de JEE que podría ser un lanzador de la librería para su uso dado que java es mas popular con JEE que con JSE.