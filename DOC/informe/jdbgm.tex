\chapter{JDBGM}
%
%
En este capitulo se introducirá formalmente el proyecto propuesto por este trabajo, se empezara por recordar la problemática expuesta en el capitulo \ref{capitulo:intro} pero de una manera más técnica, además se enmarcara el proyecto en un escenario más especifico definiendo el lenguaje sobre el cual se desarrollara y los \dd a los que se les dará soporte, para luego finalmente exponer la solución propuesta.
%
\section{El problema}
%
Como se empezó a describir en la introducción del capitulo \ref{capitulo:intro} una de las dificultades que se pueden encontrar al usar motores de bases de datos en el desarrollo de sistemas informáticos se traduce en problemas asociados a la dependencia existente con el uso de un motor de base de datos en particular, estos problemas se pueden ver genéricamente como problemas de mantenibilidad y portabilidad:
%
\begin{itemize}
%
\item Problemas en la portabilidad: Al utilizar un motor\footnote{Hablando de motores de bases de datos} en particular se esta atando al software en mayor o menor medida al uso de este según como haya sido diseñado el sistema, el mayor problema al que hay enfrentarse se traduce en las diferentes sintaxis para SQL que define cada \dd. Culpa de ello es muy probable que una sentencia valida para un motor no lo sea para otro, por lo tanto a la hora de querer migrar desde un motor a otro e inclusive a una versión más nueva del mismo motor es necesario  ``actualizar'' las sentencias para que se apeguen a la sintaxis del motor al que se pretende migrar. Además como parte de su sintaxis particular cada \dd define sus propios tipos de datos lo que añade un poco más de dificultad al proceso de cambio o migración desde un motor a otro.  
%
\item  Problemas en el mantenimiento: al utilizar bases de datos es importante como se diseña el acceso y manejo de los \dd, hay que tener en cuenta las siguientes cuestiones: quien se tiene que hacer cargo de la persistencia de los datos? El programa tiene que conocer los datos o más bien como obtener los datos? Estas cuestiones no son nuevas y se solucionan en parte siguiendo patrones de diseño. Un ejemplo más conciso de esto lo podemos ver cuando se esta diseñando un modulo de un sistema que precisa persistir ciertos datos en una BD, la pregunta que hay que hacerse es quien debería hacerse cargo de realizar esta persistencia? Si lo hace el modulo en si se encontrarían accesos a la base de datos mezclados con la lógica de negocio del modulo, además estos accesos implican que el modulo debe conocer como conectarse con la base de datos y de que modo se deben enviar y recibir los datos, con lo que el modulo	perdería cohesión y ganaría complejidad derivando todo esto en una mayor dificultad en el mantenimiento.
\end{itemize}
%
Estos dos puntos están fuertemente relacionados pues la mantenibilidad del software esta muy ligada a la portabilidad y viceversa, pero cuando se habla de que el software sea mantenible se esta hablando de muchos aspectos más.

Este trabajo se centrara en el uso de la base de datos y en como esto afecta a la mantenibilidad del software. Como ya se dijo es de buena practica el uso de patrones de diseño, así que en la sección siguiente  se explicara brevemente que son los patrones y como estos van a ayudar a encontrar una base de diseño para el proyecto en el que se esta trabajando. 
%
%
\section{Patrones de diseño}
%
Un patrón de diseño es una solución genérica y reusable  a un problema que ocurre de manera frecuente en un contexto dado. Un patrón de diseño no es un diseño terminado que pueda ser transformado directamente en código es más bien una guía que indica como resolver un problema en determinados escenarios. Así los patrones de diseño son buenas costumbres que uno mismo debe implementar en la aplicación a desarrollar y en este mismo sentido son recomendaciones a tener en cuenta y no obligaciones\cite{Metsker:2002:DPJ}. Además es importante notar que el uso de patrones no garantiza éxito a la hora de diseñar, la descripción de un patrón indica cuando este puede ser aplicable, pero solo la experiencia indicara adecuadamente cuando el uso de un patrón de diseño en particular mejorara el diseño del software\cite{java:patrones}.

Entonces por que tener en cuenta a los patrones?
\begin{itemize}
\item Han sido probados. Los patrones reflejan la experiencia, conocimiento y perspectiva de desarrolladores quienes han aplicado satisfactoriamente estos patrones en su propio trabajo.
%
\item Son reusables. Los patrones proveen soluciones ya descubiertas que pueden ser aplicadas a diferentes problemas.
%
\item Son expresivos. Los patrones proveen un vocabulario común de soluciones que pueden expresar soluciones extensas de manera concisa.
%They are expressive. Patterns provide a common vocabulary of solutions that can express large solutions succinctly.
\end{itemize}

Existe mucha teoría sobre los patrones pero en este trabajo no interesa ahondar sobre ellos, si no que interesa introducir el concepto al lector para poder presentar de mejor manera un patrón de diseño que servirá de base para el desarrollo de \jj.
%
\subsection{Data Acces Object}\label{jdbgm:dao}
Antes de describir este patrón es necesario ubicarse en un \textbf{contexto}, al acceso a los datos depende de la fuente de datos con la que se esté trabajando. El acceso a almacenamiento persistente de datos, tal como una base de datos, varia fuertemente dependiendo del tipo de almacenamiento (BD relacionales, BD orientadas a objetos, archivos planos, etc) y de la implementación de un proveedor en particular.

El \textbf{problema} se da por que en cierto punto las aplicaciones necesitan persistir sus datos. Para muchas aplicaciones, la persistencia de los datos es implementada mediante diferentes mecanismos y hay marcadas diferencias en las API's usadas para acceder a estos diferentes mecanismos. Otras aplicaciones quizás necesite acceder a datos almacenados en diferentes sistemas muy distintos del cual se esta trabajando los cuales exigen utilizar sus API's las cuales usualmente son propietarias (no se puede acceder a el código de la misma). Esta disparidad entre las diferentes fuentes de datos produce desafíos en el diseño y además crea una potencial dependencia directa entre el código de la aplicación y el código de acceso a los datos. Dicha dependencia en el código de los componentes de la aplicación vuelve tediosa y dificultosa la migración desde un tipo de persistencia de datos a otra pues cuando cambia la fuente de datos el componente debe ser modificado para poder manejar la nueva fuente de datos.

La \textbf{solución} a este problema viene por usar un DAO (Data Acces Object) para abstraer y encapsular todos los accesos a la fuente de datos. DAO maneja la conexión con la fuente de datos para obtener y almacenar los datos.

EL DAO implementa los mecanismos de acceso necesarios para trabajar con la fuente de datos, esta fuente de datos puede ser un almacén de persistencia de cualquier tipo como por ejemplo un RDBMS, un servicio externo, un repositorio o incluso archivos xml. El componente del negocio (aquel que trabaja con la lógica del negocio) que se apoya en el DAO accede a la interfaz simplificada que este brinda para sus clientes. El DAO oculta completamente los detalles de la implementación de la fuente de datos a sus clientes y como la interfaz expuesta a los clientes del DAO no cambia cuando cambia la implementación de la fuente de datos este patrón permite que el DAO se adapte a diferentes esquemas de almacenamiento sin afectar sus clientes o componentes del negocio. Esencialmente el DAO actúa como un adaptador entre el componente y la estructura de la fuente de datos.

% The DAO completely hides the data source implementation details from its clients. Because the interface exposed by the DAO to clients does not change when the underlying data source implementation changes, this pattern allows the DAO to adapt to different storage schemes without affecting its clients or business components. Essentially, the DAO acts as an adapter between the component and the data source.
%
La figura \ref{fig:dao-structure} muestra un diagrama de clases que muestras las relaciones en el patrón DAO.  

\begin{figure}[h]
  \centering
    \includegraphics[width=0.6\textwidth]{figuras/dao-structure.jpg}
  \caption{Data Acces Object}
  \label{fig:dao-structure}
\end{figure}

Para finalizar se listaran algunas de las consecuencias del uso de este patrón:
\begin{itemize}
\item Provee transparencia. El acceso a la fuente de datos es transparente pues los detalles de la implementación están ocultos en el DAO. 
%
\item Facilita la migración.%Enables Easier Migration A layer of DAOs makes it easier for an application to migrate to a different database implementation. The business objects have no knowledge of the underlying data implementation. Thus, the migration involves changes only to the DAO layer. Further, if employing a factory strategy, it is possible to provide a concrete factory implementation for each underlying storage implementation. In this case, migrating to a different storage implementation means providing a new factory implementation to the application. 
%
\item Reduce la complejidad del código en los objetos que manejan la lógica del negocio. %Reduces Code Complexity in Business ObjectsBecause the DAOs manage all the data access complexities, it simplifies the code in the business objects and other data clients that use the DAOs. All implementation-related code (such as SQL statements) is contained in the DAO and not in the business object. This improves code readability and development productivity. 
%
\item Centraliza todo el acceso a los datos en una capa separada. %Centralizes All Data Access into a Separate Layer Because all data access operations are now delegated to the DAOs, the separate data access layer can be viewed as the layer that can isolate the rest of the application from the data access implementation. This centralization makes the application easier to maintain and manage. 
%
\item Agrega una capa extra, la cual debe ser diseñada e implementada para beneficiarse del uso de este patrón. %Adds Extra Layer The DAOs create an additional layer of objects between the data client and the data source that need to be designed and implemented to leverage the benefits of this pattern. But the benefit realized by choosing this approach pays off for the additional effort. 
%
\item Necesita diseño de jerarquía de clases, que implica otro esfuerzo extra. %Needs Class Hierarchy Design When using a factory strategy, the hierarchy of concrete factories and the hierarchy of concrete products produced by the factories need to be designed and implemented. This additional effort needs to be considered if there is sufficient justification warranting such flexibility. This increases the complexity of the design. However, you can choose to implement the factory strategy starting with the Factory Method pattern first, and then move towards the Abstract Factory if necessary.
%
\end{itemize}
%
%
\section{La solución propuesta: \jj}
Después de conocer el anterior patrón se puede inferir que el problema al que apunta este trabajo es común y a veces inevitable dependiendo de las necesidades especificas del sistema con el que se esté tratando, entonces por que desarrollar este proyecto? Pues para aquellos desarrollos donde el patrón sea aplicable donde, por ejemplo, los tiempos de respuesta no sean un factor critico recordando que DAO agrega una capa extra entre los datos y la lógica del negocio lo que en el fondo significa un tiempo de acceso un poco mayor comparado a un acceso directo, otro caso donde la solución puede ser aplicable es cuando se debe trabajar con diferentes fuentes de datos al mismo tiempo para por ejemplo persistir datos en diferentes motores al mismo tiempo. Tema aparte es que no se estará contemplando completamente la especificación de DAO si no que se lo estará limitando a bases de datos relacionales. Entonces con las aclaraciones dadas, a continuación se detallara de que trata el proyecto.

Para trabajar con \dd Java provee JDBC\cite{java:jdbc} que según su documentación provee un medio para acceder virtualmente a cualquier tipo de datos tabulares. JDBC es usualmente usado para enviar comandos SQL directamente hacia el motor, aunque en realidad fue diseñado para ser la base sobre la cual se construyen herramientas e interfaces alternativas más ``amigables con el programador'' en el sentido que se puede construir un API más entendible o conveniente para un proyecto en particular que en el fondo es ``traducida'' a \jd. Como en esté proyecto se pretende ocultar el acceso explicito a una base de datos no se puede usar directamente \jd pues al hacerlo se necesita explicitar la conexión con el motor de la base de datos desde el componente en donde se esté realizando la persistencia de los datos, por otro lado las sentencias SQL deben ser pasadas como cadenas de texto lo que como ya fue expuesto presenta cierta dependencia con un motor en particular al estar tratando con tipos de datos inmutables (la sentencia SQL como \textit{string}) que deben ser cambiados cuando se quiera alterar la sentencia.

Entonces haciendo uso de la idea de DAO se puede expresar lo siguiente sobre \jj: ``Un adaptador entre la capa (o componentes) del negocio y la base de datos relacional subyacente que oculta los detalles específicos sobre el uso de un \dd en particular'', también se puede decir que \jj sera, en un diseño en capaz, parte esencial de la capa de acceso a datos. Se presentan las siguientes características básicas para el proyecto:
%
\begin{itemize}
\item Debe encargarse de gestionar la conexión hacia la base de datos.
\item Debe Eliminar o limitar la dependencia del uso de los dialectos SQL.
\item Debe proveer una estructura fácilmente ampliable pues inicialmente el proyecto tendrá un soporte limitado en cuanto motores de base de datos.
\end{itemize}
%
El objetivo primario de \jj es ofrecer un medio accesible para independizarse de el uso de un motor en particular y como consecuencia de ello se promueve el uso de una estructura ampliamente probada (basado en DAO). Hay algunos aspectos que no se pueden evitar, por ejemplo si queremos migrar de motor de base de datos es necesario que anteriormente la base se haya migrado al nuevo motor pues con lo que esta lidiando el proyecto es con el acceso a la base de datos no con la administración de la base de datos, otro aspecto es que para poder acceder a la base de datos es necesario conocer su URI\footnote{\textit{Uniform Resource Identifier}} y nombre de usuario y contraseña usados para acceder al motor. En el capitulo siguiente se dará una descripción más detallada de los requisitos necesarios para cumplir con los objetivos que persigue el proyecto. 